\begin{frame}{Introduction}

\begin{itemize}
    \item To defend the potential threat from large-scale quantum computers, the US National Institute of Standards and Technology (NIST) initiated standardization process for post-quantum cryptography in 2016.
    \pause
    \item In 2022, four selected algorithms – CRYSTALS-Kyber, CRYSTALS-Dilithium, \textsc{Falcon}, and SPHINCS+ were expected to be part of NIST's post-quantum cryptographic standards.
\end{itemize}
    
\end{frame}


\begin{frame}{Theoretical Security - Hardness of Mathematical Problems}

In theory, these algorithms can base their security on problems that are considered still hard given the advantage of quantum computing.

\begin{enumerate}
    \item CRYSTALS-Kyber: Module Learning With Errors (MLWE)
    \item CRYSTALS-Dilithium: Module Short Integer Solution (MSIS)
    \item \textsc{Falcon}: NTRU Problem and SIS on NTRU lattices
    \item SPHINCS+: Security of the used hash function families
\end{enumerate}

\end{frame}


\begin{frame}{Real-World Security – Side-Channel Attacks}

In practice, the implementations of these algorithms can suffer side-channel attacks. Fortunately, there are countermeasures for them.

\begin{enumerate}
    \item CRYSTALS-Kyber: \cite{TCHES:BGRSv21, TCHES:FBRKSVS22, EPRINT:HKLPSS22}
    \item CRYSTALS-Dilithium: \cite{ACNS:MGTF19}
    \item {\color<2>{red} \textsc{Falcon}: \cite{PQCRYPTO:HPRR20, TCHES:GMRR22, EC:ZLYW23} }
    \item SPHINCS+: \cite{bertoni2010building,IEEE:7223170}
\end{enumerate}
\pause
\medskip

Unfortunately, there are attacks on \textsc{Falcon} that have not been addressed.

\end{frame}


\begin{frame}{Attacks on \textsc{Falcon}}

%\begin{figure}
%    \centering
%    \includegraphics[width=\textwidth]{tikz/falcon-sign.png}
%    \caption{A graphical overview of {\sf FALCON.Sign}}
%    \label{fig:enter-label}
%\end{figure}

\begin{figure}
\centering

% \begin{tikzpicture}[node distance=2cm, gridded]
\begin{tikzpicture}[node distance=2cm]

% Gadgets
\node (hash) [rectangle, rounded corners, 
minimum width=1cm, 
minimum height=1cm,
text centered,
fill=white!70!black,
draw=black] at (0,0) {$H$};
\coordinate (hash-in1) at ($(hash) + (-0.5,0.2)$);
\coordinate (hash-in2) at ($(hash) + (-0.5,-0.2)$);


\node (FFT) [rectangle, rounded corners, 
minimum width=1cm, 
minimum height=1cm,
text centered,
fill=white!70!black,
draw=black, right of=hash, xshift=-0.5cm] {{\sf FFT}}; 

\node (point-mult) [rectangle, rounded corners, 
minimum width=1cm, 
minimum height=1cm,
text centered, 
fill=white!50!red,
draw=black, right of=FFT, xshift=-0.5cm] {$\bigodot$};
\coordinate (point-mult-in1) at ($(point-mult) + (-0.5,0.2)$);
\coordinate (point-mult-in2) at ($(point-mult) + (-0.5,-0.2)$);

\node (samp) [rectangle, rounded corners, 
minimum width=1cm, 
minimum height=1cm,
text centered, 
fill=white!50!green,
draw=black, right of=point-mult] {{\sf Samp}};
\coordinate (samp-in1) at ($(samp) + (-0.5,0.2)$);
\coordinate (samp-in2) at ($(samp) + (-0.5,-0.2)$);

\node (compare) [rectangle, rounded corners, 
minimum width=1cm, 
minimum height=1cm,
text centered, 
fill=white!70!black,
draw=black, right of=samp] {{\sf Compare}};

\node (invFFT) [rectangle, rounded corners, 
minimum width=1cm, 
minimum height=1cm,
text centered, 
fill=white!70!black,
draw=black, right of=compare] {${\sf invFFT}$};
\coordinate (invFFT-out1) at ($(invFFT.east) + (0,0.2)$);
\coordinate (invFFT-out2) at ($(invFFT.east) + (0,-0.2)$);

\node (compress) [rectangle, rounded corners, 
minimum width=1cm, 
minimum height=1cm,
text centered, 
fill=white!70!black,
draw=black, right of=invFFT] {{\sf Compress}};


% Texts
\node (m) at ($(hash-in1) + (-0.75,0)$) {{\sf m}};
\node (r) at ($(hash-in2) + (-0.75,0)$) {{\sf r}};
% \node (B) at ($(point-mult-in1) + (-1.5,0)$) {$\mathbf{\hat B}$};
\node (sk) at ($(point-mult.north) + (0,1)$) {$\sf sk$};
% \node (beta) at ($(compare.north) + (0,1)$) {$\lfloor \beta^2 \rfloor$};
% \node (s1) at ($(invFFT.east) + (0.4,1)$) {$s_1$};
\node (s) at ($(compress.east) + (0.75,0)$) {$\sf s$};

% Lines
\draw [arrow] (m) -- (hash-in1);
\draw [arrow] (r) -- (hash-in2);
\draw [arrow] (hash.east) -- node[above]{} (FFT.west);

\draw [arrow] (FFT.east) -- (point-mult.west);
\draw [arrow] (sk) -- node[left]{} (point-mult.north);

\draw [arrow] (point-mult.east) -- node[above]{} (samp.west);
\draw [arrow] (sk) -| node[above left]{} (samp.north);

\draw [arrow] (samp.east) -- (compare.west);

% \draw [arrow] (beta) -- (compare.north);

\draw [arrow] (compare.east) -- (invFFT.west);

\draw [arrow] (invFFT.east) -- (compress.west);
% \draw [arrow] (invFFT.east) -| node[below right]{$s_2$} (s1);

\draw [arrow] (compress.east) -- (s);


\end{tikzpicture}

\caption{A graphical overview of {\sf FALCON.Sign}.}
\label{fig:falcon-sign-test}
\end{figure}


\begin{center}
{\small
\begin{tabular}{ l | c | c }
 & Attack & Countermeasure \\
\hline
\makecell{{\color{red}Pre-image Vector Computation}} & \cite{KA21, TCHES:GMRR22} & \textcolor<2>{red}{None} \\
\hline
\makecell{{\textcolor{black!30!green}{Gaussian Sampler over Lattices}}} & \cite{TCHES:GMRR22, EC:ZLYW23} & \cite{TCHES:GMRR22, EC:ZLYW23} \\ 
\end{tabular}
}
\end{center}
    
\end{frame}


\begin{frame}{Our Contributions}

In this paper, we present the following contributions:
\pause
\begin{itemize}
    \item We propose the first masking scheme on the floating-point number multiplication and addition in the pre-image vector computation of \textsc{Falcon} as a countermeasure.
    \pause
    \item We verify the high-order security of our design in the probing model.
    \pause
    \item To test the practical leakage of our work, we conduct the Test Vector Leakage Assessment (TVLA) \cite{gilbert2011testing} experiments.
    \pause
    \item We also test the performance by comparing with the reference implementation of \textsc{Falcon} \cite{NISTPQC-R3:FALCON20}.
\end{itemize}


\end{frame}


\begin{frame}{Notation}

Throughout the presentation, we assume
\pause
\begin{itemize}
%    \item $M > N$ are two positive integers and $N = 2^\kappa$ for some integer $\kappa$.
%    \pause
%    \item $\phi = x^N + 1$ is a polynomial.
%    \pause
%    \item $q$ is a prime number.
%    \pause
%    \item Represent a vector $\mathbf{v}$ by a boldface letter, and a matrix $\mathbf{A}$ by a boldface capital letter.
%    \item For a polynomial $f \in \Z[x] / \phi$, it can be considered as an $N$-by-$N$ matrix.
    \item For a variable $x$, the $j$th bit of $x$ is written as $x^{(j)}$.
    \pause
    \item The $i$th bit to $j$th bit ($j \geq i$) of $x$ is represented by $x^{[j:i]}$.
    \pause
    \item A sequence of $n$ variables $(x_1, x_2, \cdots, x_n)$ (e.g. shares of variable $x$) is written as $(x_i)_{1 \leq i \leq n}$, or simply $(x_i)$.
    \pause
    \item For a proposition $P$, $\llbracket P \rrbracket = 1$ if and only if $P$ is true and $0$ if otherwise.
\end{itemize}
    
\end{frame}
