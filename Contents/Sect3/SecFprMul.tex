\ifhandout
\else
\ifdetailed

\begin{frame}{SecFprMul: Secure FprMul}

\centerline{
\blockalgstart{.9\textwidth}{{\large FprMul}}
\begin{algorithm}[H]
  \algsetup{linenosize=\small}
  \begin{multicols}{2}
  \begin{algorithmic}[1]
    \REQUIRE \textcolor<2>{red}{FPN $x = (sx, ex, mx)$}
    \REQUIRE \textcolor<2>{red}{FPN $y = (sy, ey, my)$}
    \ENSURE FPN product of $x$ and $y$

    \STATE \textcolor<3>{red}{ $s \gets sx \oplus sy$}
    \STATE \textcolor<3>{red}{ $e \gets ex + ey - 2100$}

    \STATE \textcolor<4>{red}{$z \gets mx \times my$}

    \STATE \textcolor<5-6>{red}{ $b \gets \llbracket z^{[50:1]} \neq 0 \rrbracket$}
    \STATE \textcolor<5-6>{red}{ $z \gets z^{[106:51]} \vee b$ }
    \STATE \textcolor<5-6>{red}{ $z' \gets (z \gg 1) \vee z^{(1)}$ }
    \STATE \textcolor<5-6>{red}{ $w \gets z^{(106)}$ }
    \STATE \textcolor<5-6>{red}{$z \gets z \oplus (z \oplus z') \wedge (-w)$ }

    \STATE \textcolor<7>{red}{$e \gets e + w$}
 
    \STATE \textcolor<8>{red}{$bx \gets \llbracket ex \neq 0 \rrbracket$, $by \gets \llbracket ey \neq 0 \rrbracket$}
    \STATE \textcolor<8>{red}{$b \gets bx \wedge by$}
    \STATE \textcolor<8>{red}{$z \gets z \wedge (-b)$} \\
    \RETURN \hskip -4pt \textcolor<9>{red}{${\sf FPR}(s, e, z)$}
  \end{algorithmic}
  \end{multicols}
\vspace{-10pt}
\end{algorithm}

\blockalgend
}
\medskip

\only<1>{
We show how we mask the floating-point number multiplication algorithm {\sf FprMul}.
}

\only<2>{
We assume $(sx_i)$ and $(sy_i)$ are Boolean shares, $(ex_i)$ and $(ey_i)$ are 16-bit arithmetic shares, and $(mx_i)$ and $(my_i)$ are 128-bit arithmetic shares, which can load the product of two $53$-bit values.
}


\only<3>{
These can be operated share-wise.
}


\only<4>{
This is done by {\sf SecMult}. For further operations, we then apply an {\sf A2B} to turn them to Boolean shares.
}


\only<5>{
Conditional shift by $50$ bits and $51$ bits, depending on $z^{(106)}$, while preserving the sticky bit.

These can be done by {\sf SecNonzero} and {\sf SecOr}.

}

\only<6>{
We observe that we can save one {\sf SecOR}.
\begin{itemize}
\item When shifted by $50$ bits, we OR the last bit with $z^{[50:1]}$.
\item When shifted by $51$ bits, we OR the last bit with $z^{[51:1]}$.
\end{itemize}
We can simply OR the the last bit with $z^{[51:1]}$, regardless of the conditional shift result.
}


\only<7>{
This is by adding to any share.
}



\only<8>{
This is by {\sf SecNonzero} and {\sf SecAnd}, and applying the simple trick.
}



\only<9>{
Now it calls {\sf FPR} to return a 64-bit Boolean-masked FPN.
}

\end{frame}

%
%
%
\fi %detailed

\fi %handout



\begin{frame}{SecFprMul: Secure FprMul}


\centerline{
\blockalgstart{\textwidth}{SecFprMul}
\begin{algorithm}[H]
  \algsetup{linenosize=\small}
  \label{alg:SecFprMul}
  \small
  \begin{multicols}{2}
  \begin{algorithmic}[1]
    \REQUIRE Shares ${(sx_i)_{1\leq i \leq n}}, {(ex_i)_{1\leq i \leq n}}, {(mx_i)_{1\leq i \leq n}}$
    \REQUIRE Shares ${(sy_i)_{1\leq i \leq n}}, {(ey_i)_{1\leq i \leq n}}, {(my_i)_{1\leq i \leq n}}$
    \ENSURE Boolean shares for the FPN product.
    \STATE $(s_i) \gets (sx_i \oplus sy_i)$ \label{alg:SecFprMul:sign_add}
    \STATE $(e_i) \gets (ex_1 + ey_1 - 2100, ex_2 + ey_2, \cdots)$ \label{alg:SecFprMul:exp_add}
    \STATE $(p_i) \gets {\sf SecMult}((mx_i), (my_i))$ \label{alg:SecFprMul:mant_mul}
    \STATE $(p_i) \gets {\sf A2B}((p_i))$ \label{alg:SecFprMul:128A2B}
    \STATE $(b_i) \gets {\sf SecNonzero}( (p_i^{[51:1]}) )$ \label{alg:SecFprMul:bool_nonzero}
    \STATE $(z_i) \gets (p_i^{[105:51]})$ \label{alg:SecFprMul:shift_start}
    \STATE $(z'_i) \gets (p_i^{[105:51]} \oplus p_i^{[106:52]})$
    \STATE $(w_i) \gets (p_i^{(106)})$
    \STATE $(z'_i) \gets {\sf SecAnd}( (z'_i), {\sf Refresh}((-w_i) ))$ \label{alg:SecFprMul:condition}
    \STATE $(z_i) \gets (z'_i \oplus z_i)$ \label{alg:SecFprMul:shift}
    \STATE $(z_i) \gets {\sf SecOr}( (z_i), (b_i) )$ \label{alg:SecFprMul:stickybit}
    \STATE $(w_i) \gets {\sf B2A_{Bit}}((w_i))$
    \STATE $(e_i) \gets (e_i + w_i)$ \label{alg:SecFprMul:add106}
    \STATE $(bx_i) \gets {\sf SecNonzero}( (ex_i) )$ \label{alg:SecFprMul:arith_nonzero_1}
    \STATE $(by_i) \gets {\sf SecNonzero}( (ey_i) )$ \label{alg:SecFprMul:arith_nonzero_2}
    \STATE $(d_i) \gets {\sf SecAnd}( (bx_i), (by_i) )$
    \STATE $(z_i) \gets {\sf SecAnd}( (z_i) , (-d_i^{(1)}) )$ \label{alg:SecFprMul:check_exp}
    \STATE \Return ${\sf SecFPR}( (s_i) , (e_i) , (z_i) )$
  \end{algorithmic}
  \end{multicols}
\vspace{-10pt}
\end{algorithm}

\blockalgend
}
\medskip



\end{frame}
