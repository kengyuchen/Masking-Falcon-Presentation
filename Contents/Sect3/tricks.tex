\begin{frame}{Simple Tricks}

If we want to operate the following operations:
\centerline{
\blockalgstart{.2\textwidth}{}
\begin{algorithm}[H]
  \algsetup{linenosize=\small}
  \begin{algorithmic}[1]
    \IF{$a = 0$}
        \STATE $b \gets 0$
    \ENDIF
\end{algorithmic}
\end{algorithm}
\vspace{-2pt}
\blockalgend
}
\medskip

Suppose $a$ is one-bit, we may write it as
\vskip -4pt
\centerline{
\blockalgstart{.2\textwidth}{}
\begin{algorithm}[H]
  \algsetup{linenosize=\small}
  \begin{algorithmic}[1]
    \STATE $b \gets b \wedge (-a)$
\end{algorithmic}
\end{algorithm}
\vspace{-3pt}
\blockalgend
}
\medskip

We also apply the same idea for Boolean-shared values in our design
% \vskip -2pt
\centerline{
\blockalgstart{.34\textwidth}{}
\begin{algorithm}[H]
  \algsetup{linenosize=\small}
  \begin{algorithmic}[1]
    \STATE $(b_i) \gets {\sf SecAnd}((b_i), (-a_i))$
\end{algorithmic}
\end{algorithm}
\vspace{-3pt}
\blockalgend
}
\medskip

We utilize that $\bigoplus_{i=1}^n -a_i = -a$, which is not true for general $k$-bit $a$.
\end{frame}


\begin{frame}{Simple Tricks}

Similarly, for operations
\centerline{
\blockalgstart{.2\textwidth}{}
\begin{algorithm}[H]
  \algsetup{linenosize=\small}
  \begin{algorithmic}[1]
    \IF{$a = 1$}
        \STATE $b \gets 0$
    \ENDIF
\end{algorithmic}
\end{algorithm}
\vspace{-2pt}
\blockalgend
}
\medskip

Suppose $a$ is one-bit, we may write it as
\vskip -4pt
\centerline{
\blockalgstart{.23\textwidth}{}
\begin{algorithm}[H]
  \algsetup{linenosize=\small}
  \begin{algorithmic}[1]
    \STATE $b \gets b \wedge (\neg(-a))$
\end{algorithmic}
\end{algorithm}
\vspace{-3pt}
\blockalgend
}
\medskip

For Boolean-shared values,
% \vskip -2pt
\centerline{
\blockalgstart{.45\textwidth}{}
\begin{algorithm}[H]
  \algsetup{linenosize=\small}
  \begin{algorithmic}[1]
    \STATE $(c_i) \gets (-a_i)$
    \STATE $(b_i) \gets {\sf SecAnd}((b_i), (\neg c_1, c_2, \cdots, c_n))$
\end{algorithmic}
\end{algorithm}
\vspace{-3pt}
\blockalgend
}
\medskip
\end{frame}


\begin{frame}{Simple Tricks}

Moreover, for operations,
\centerline{
\blockalgstart{.2\textwidth}{}
\begin{algorithm}[H]
  \algsetup{linenosize=\small}
  \begin{algorithmic}[1]
    \IF{$a = 1$}
        \STATE $b \gets c$
    \ENDIF
\end{algorithmic}
\end{algorithm}
\vspace{-2pt}
\blockalgend
}
\medskip

Suppose $a$ is one-bit, we may write it as
\vskip -4pt
\centerline{
\blockalgstart{.25\textwidth}{}
\begin{algorithm}[H]
  \algsetup{linenosize=\small}
  \begin{algorithmic}[1]
    \STATE $d \gets b \oplus c$
    \STATE $b \gets b \oplus (d \wedge (-a))$
\end{algorithmic}
\end{algorithm}
\vspace{-3pt}
\blockalgend
}
\medskip

For Boolean-shared values,
% \vskip -2pt
\centerline{
\blockalgstart{.35\textwidth}{}
\begin{algorithm}[H]
  \algsetup{linenosize=\small}
  \begin{algorithmic}[1]
    \STATE $(d_i) \gets (b_i \oplus c_i)$
    \STATE $(d_i) \gets {\sf SecAnd}((d_i), (-a_i))$
    \STATE $(b_i) \gets (b_i \oplus d_i)$
\end{algorithmic}
\end{algorithm}
\vspace{-3pt}
\blockalgend
}
\medskip
\end{frame}



\begin{frame}{Simple Tricks in Masking FPR}

\centerline{
\blockalgstart{.6\textwidth}{{\large FPR}}
\begin{algorithm}[H]
  \algsetup{linenosize=\small}
  \begin{algorithmic}[1]
    \REQUIRE Sign bit $s$, exponent $e$, and 55-bit mantissa $z$
    \ENSURE FPN $x$ packed by $s, e, z$
    \STATE $e \gets e + 1076$
    \STATE $b \gets \llbracket e < 0 \rrbracket$
    \STATE \textcolor{red}{$z \gets z \wedge (b-1)$}
    \STATE $b \gets \llbracket z \neq 0 \rrbracket $
    \STATE \textcolor{red}{$e \gets e \wedge (-b)$}
    \STATE $x \gets ((s \ll 63) \vee (z \gg 2)) + e \ll 52$
    \STATE $f \gets {\tt0XC8} \gg z^{[3:1]} $
    \STATE $x \gets x + f^{(1)}$ \COMMENT{increment if $z^{[3:1]}$ is 011,110 or 111} \\ 
    \RETURN \hskip -4pt $x$
  \end{algorithmic}
\end{algorithm}
\blockalgend
}

\end{frame}


\begin{frame}{Simple Tricks in Masking FprMul}

\centerline{
\blockalgstart{.9\textwidth}{{\large FprMul}}
\begin{algorithm}[H]
  \algsetup{linenosize=\small}
  \begin{multicols}{2}
  \begin{algorithmic}[1]
    \REQUIRE FPN $x = (sx, ex, mx)$
    \REQUIRE FPN $y = (sy, ey, my)$
    \ENSURE FPN product of $x$ and $y$
    \STATE $s \gets sx \oplus sy$ 
    \STATE $e \gets ex + ey - 2100$ 
    \STATE $z \gets mx \times my$
    \STATE $b \gets \llbracket z^{[50:1]} \neq 0 \rrbracket$
    \STATE $z \gets z^{[106:51]} \vee b$
    \STATE $z' \gets (z \gg 1) \vee z^{(1)}$
    \STATE $w \gets z^{(106)}$
    \STATE \textcolor{red}{$z \gets z \oplus (z \oplus z') \wedge (-w)$}
    \STATE $e \gets e + w$
    \STATE $bx \gets \llbracket ex \neq 0 \rrbracket$, $by \gets \llbracket ey \neq 0 \rrbracket$
    \STATE $b \gets bx \wedge by$
    \STATE \textcolor{red}{$z \gets z \wedge (-b)$} \\
    \RETURN \hskip -4pt ${\sf FPR}(s, e, z)$
  \end{algorithmic}
  \end{multicols}
\end{algorithm}
\vspace{-10pt}
\blockalgend
}

\end{frame}


\begin{frame}{Simple Tricks in Masking FprAdd}

\centerline{
\blockalgstart{.9\textwidth}{{\large FprAdd}}
\begin{algorithm}[H]
  \algsetup{linenosize=\small}
  \begin{multicols}{2}
  \small
  \begin{algorithmic}[1]
    \REQUIRE FPNs $x$ and $y$
    \ENSURE FPN sum of $x$ and $y$
    \STATE $d \gets x^{[63:1]} - y^{[63:1]}$
    \STATE $cs \gets d^{(64)} \vee ((1 - (-d)^{(64)}) \wedge x^{(64)}) $
    \STATE \textcolor{red}{$m \gets (x \oplus y) \wedge (-cs)$}
    \STATE \textcolor{red}{$x \gets x \oplus m, y \gets y \oplus m$}
    \STATE Extract $(sx, ex, mx)$ and $(sy, ey, my)$ from $x, y$, respectively.
    \STATE $mx \gets mx \ll 3, my \gets my \ll 3$
    \STATE $ex \gets ex - 1078, ey \gets ey - 1078$
    \STATE $c \gets ex - ey$
    \STATE $b \gets \llbracket c < 60 \rrbracket$ 
    \STATE \textcolor{red}{$my \gets my \wedge (-b) $ }
    \STATE $my \gets (my \gg c) \vee \llbracket my^{[c:1]} \neq 0 \rrbracket$
    \STATE $s \gets sx \oplus sy$
    \STATE $z \gets mx + (-1)^s my $
    \STATE Normalize $z, ex$ to make the 64th bit of $z$ set
    \STATE $z \gets (z \gg  9) \vee \llbracket z^{[9:1]} \neq 0 \rrbracket$
    \STATE $ex \gets ex + 9$ \\
    \RETURN \hskip -4pt ${\sf FPR}(sx, ex, z)$
  \end{algorithmic}
  \end{multicols}
\end{algorithm}
\vspace{-10pt}
\blockalgend
}

\end{frame}
