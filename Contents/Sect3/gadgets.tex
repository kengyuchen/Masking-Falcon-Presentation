\begin{frame}{Overview of Our Approach}

We now show how we mask {\sf FPR}, {\sf FprMul}, and {\sf FprAdd}.
\pause

An intuitive approach to mask an algorithm:
\pause

\begin{itemize}
\item For operations like $\wedge, \oplus$: Boolean masking
\pause
\item For operations like $+, \times$: arithmetic masking
\end{itemize}
\pause
and use the following gadgets if necessary:
\pause
\begin{itemize}
	\item {\sf A2B}: $(x_i)_{1\leq i \leq n} \mapsto (y_i)_{1\leq i \leq n}$ such that $\sum_{i=1}^n x_i = \bigoplus_{i=1}^n y_i$
	\pause
	\item {\sf B2A}: $(y_i)_{1\leq i \leq n} \mapsto (x_i)_{1\leq i \leq n}$ such that $\bigoplus_{i=1}^n y_i = \sum_{i=1}^n x_i$
\end{itemize}

\end{frame}


\begin{frame}{Overview of Our Approach}

For example, if $x$ is a secret variable, the operation $y \gets {\sf pt} \oplus x; \quad z \gets y \times x$ will become
\pause

\centerline{
	\begin{minipage}{.65\textwidth}
	\begin{block}{}
	\vskip -8pt
	\begin{algorithm}[H]
	\algsetup{linenosize=\small}
	\small
	\begin{algorithmic}[1]
		\REQUIRE ${\sf pt}, x$ 
		\ENSURE $(z_1, z_2)$ such that $z_1 + z_2 = ({\sf pt} \oplus x) \times x$
        \STATE Sample $x_1,x_2 \getsdollar \{0,1\}^*$ such that $x_1 \oplus x_2 = x$
		\STATE $y_1 \gets {\sf pt} \oplus x_1$ 
		\STATE $y_2 \gets x_2$
		\STATE $(x_1', x_2') \gets {\sf B2A}((x_1, x_2))$ \quad // $x_1'+x_2' = x_1 \oplus x_2 = x$ 
		\STATE $(y_1', y_2') \gets {\sf B2A}((y_1, y_2))$ \quad // $y_1'+y_2' = y_1 \oplus y_2 = y$
		\STATE $z_1 \gets y_1 \times x_1 + y_2 \times x_2$
		\STATE $z_2 \gets y_1 \times x_2 + y_2 \times x_1$
    \end{algorithmic}
	\end{algorithm}

	\end{block}
	\end{minipage}
}
\medskip



\end{frame}


\begin{frame}{Overview of Our Approach}

However, some operations in floating-point number arithmetic cannot be easily implemented in this way:
\pause
\begin{itemize}
	\item Check whether a secret value is nonzero
	\begin{itemize}
		\item Given $(x_i)$, check whether $\bigoplus_{i=1}^n x_i \neq 0$ or $\sum_{i=1}^n x_i \neq 0$
	\end{itemize}
	\pause
	\item Right-shift a secret value by another secret value
	\begin{itemize}
		\item Given $(x_i)$ and $(c_i)$, right-shift $(x_i)$ by $(c_i)$
	\end{itemize}
	\pause
	\item Normalize a secret value to $[2^{63},2^{64})$
	\begin{itemize}
		\item Given $(x_i)$, left-shift $(x_i)$ until its 64th bit is set
	\end{itemize}
\end{itemize}

\end{frame}


\begin{frame}{Overview of Our Approach}

%\only<2>{Checking whether a secret value is nonzero:}
%\only<3>{Right-shifting a secret value by another secret value:}
%\only<4>{Normalizing a secret value to $[2^{54},2^{55})$:}
%\medskip
Let's see where these operations are:
\begin{itemize}
	\item \alt<2>{\textcolor{red}{Check whether a secret value is nonzero}} {Check whether a secret value is nonzero}
	\item \alt<3>{\textcolor{red}{Right-shift a secret value by another secret value}}{ Right-shift a secret value by another secret value }
	\item \alt<4>{\textcolor{red}{Normalize a secret value to $[2^{63},2^{64})$}}{Normalize a secret value to $[2^{63},2^{64})$}
\end{itemize}


\begin{columns}[T]
\column{0.5\textwidth}
FprMul:
\begin{enumerate}
    \item {\color<2->{trans}{Sign bit XOR}}
    \item {\color<2->{trans}{Exponent Addition}}
    \item {\color<2->{trans}{Mantissa Multiplication}}
    \item \alt<1,3->{ {\color<3->{trans} Right-shifting the result} }{ {\color{red}Right-shifting the result} }
    \item \alt<1,3->{ {\color<3->{trans}Packing and rounding (FPR)} }{\color{red}Packing and rounding (FPR)}
\end{enumerate}

\column{0.5\textwidth}
FprAdd:
\begin{enumerate}
    \item {\color<2->{trans} Making the first operand $\geq$ the second}
    \item \alt<1,4->{ {\color<4->{trans} Right-shifting the second operand} }{ {\color{red} Right-shifting the second operand} }
    \item { \color<2->{trans}{Mantissa Addition / Subtraction} }
    \item \alt<1,3>{ {\color<3>{trans} Normalizing the result} }{ {\color{red}Normalizing the result} }
    \item \alt<1,3->{\color<3->{trans}Packing and rounding (FPR) }{\color{red}Packing and rounding (FPR)}
\end{enumerate}

\end{columns}
\medskip


\end{frame}


\begin{frame}{Overview of Our Approach}

We design novel gadgets for these three operations, including:
\only<2->{ \begin{itemize} }
	\only<2->{ \item {\sf SecNonzero}: securely check whether a secret value is nonzero}

	\only<3->{ \item {\sf SecFprUrsh}: securely right-shift a secret value by another secret value}
	
	\only<4->{ \item {\sf SecFprNorm64}: securely normalize a secret value to $[2^{63}, 2^{64})$}
\only<2->{ \end{itemize} }
\medskip

\only<2>{
	\begin{block}{SecNonzero}
		Given shares $(x_i)$, output one-bit shares $(b_i)$ such that
		\[
			\left \llbracket \bigoplus_{i=1}^n x_i \neq 0 \right\rrbracket = \bigoplus_{i=1}^n b_i \quad \text{or} \quad \left \llbracket \sum_{i=1}^n x_i \neq 0 \right\rrbracket = \bigoplus_{i=1}^n b_i
		\]
	\end{block}
}

\only<3>{
	\begin{block}{SecFprUrsh}
		Given 64-bit shares $(x_i)$ and 6-bit $(c_i)$, output shares $(z_i)$ such that
		\[
			\bigoplus_{i=1}^n z_i = \left( \left( \bigoplus_{i=1}^n x_i \right) \gg \left( \sum_{i=1}^n c_i \bmod 2^{6} \right) \right) \vee \left \llbracket \bigoplus_{i=1}^n x_i^{[c:1]} \neq 0 \right \rrbracket
		\]
	\end{block}
}

\only<4>{
	\begin{block}{SecFprNorm64}
		Given 64-bit shares $(x_i)$ and 16-bit shares $(e_i)$, output shares $(x_i')$ and $(e_i')$ such that
		\[
			\textbf{if } c \text{ is the smallest integer such that } \left( (\oplus_{i=1}^n x_i) \ll c \right) \in [2^{63}, 2^{64})
		\]
		\[
			\textbf{then } (\oplus_{i=1}^n x_i') = \left( (\oplus_{i=1}^n x_i) \ll c \right) \text{ and } \sum_{i=1}^n e_i' = (\sum_{i=1}^n e_i) - c
		\]
	\end{block}
}

\end{frame}


\begin{frame}{Gadgets from Previous Works}

\begin{table}
\centering
	\begin{tabular}{ l l c} 
	\toprule
	\textbf{Algorithm} & \textbf{Description} & \textbf{Reference} \\
	\midrule
	{\sf SecAnd} & AND of Boolean shares & \cite{C:IshSahWag03, CCS:BBDFGS16} \\
	{\sf SecMult} & Multiplication of arithmetic shares & \cite{C:IshSahWag03, CCS:BBDFGS16} \\
	{\sf SecAdd} & Addition of Boolean shares & \cite{FSE:CGTV15, EC:BBEFGR18} \\
	{\sf A2B} & Arithmetic to Boolean conversion & \cite{PKC:SPOG19} \\
	{\sf B2A} & Boolean to arithmetic conversion & \cite{TCHES:BetCorZei18} \\
	${\sf B2A_{Bit}}$ & One-bit {\sf B2A} conversion & \cite{PKC:SPOG19} \\
	{\sf RefreshMasks} & $t$-NI refresh of masks & \cite{CCS:BBDFGS16, TCHES:BetCorZei18} \\
	{\sf Refresh} & $t$-SNI refresh of masks & \cite{CCS:BBDFGS16} \\
	\bottomrule
	\end{tabular}
	\caption{List of used gadgets in our work}
	\label{table:gadgets}
\end{table}

\end{frame}
