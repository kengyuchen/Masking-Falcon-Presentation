\begin{frame}{Fast-Fourier Transform}

The pre-image vector computation includes polynomial multiplications
\[
\textbf{t} = \left[ \begin{array}{c|c} c & 0 \\ \end{array} \right] \cdot \mathbf{B}^{-1}
= \frac{1}{q} \left[ \begin{array}{c|c} c \cdot -F & c \cdot f \\ \end{array} \right]
\]
\pause

To speed up and apply the fast Fourier nearest plane algorithm, the pre-image vector computation is performed in the Fourier domain:
\[
\frac{1}{q} \left[ \begin{array}{c|c} {\sf FFT}(c) \odot {\sf FFT}(-F) & {\sf FFT}(c) \odot {\sf FFT}(f) \end{array} \right]
\]
\pause

Therefore, the pre-image vector computation is essentially coefficient-wise complex number multiplications.

\end{frame}


\begin{frame}{Floating-Point Number}

A complex number is represented by two 64-bit floating-point numbers (FPNs). An FPN is composed of sign bit $s$, exponent $e$, and mantissa $\tilde{m}$

\begin{figure}
    \centering
    \includegraphics[width = 300pt]{Figure/fpu.png}
    \caption{A 64-bit Floating-Point Number}
    \label{fig:64bitfpr}
\end{figure}

The value is $(-1)^s \cdot 2^{e - 1023} \cdot \underbrace{(1 + \tilde{m} \cdot 2^{-52})}_{\times 2^{52} = m}$

For convenience, we may use $(s, e, m)$ to represent an FPN.

\end{frame}


\begin{frame}{Floating-Point Number Arithmetic}

\begin{columns}[T]
\column{0.5\textwidth}
FPN multiplication (FprMul) is proceeded by
    \only<2->{\begin{enumerate} \item Sign bit XOR}
    \only<3->{\item Exponent Addition}
    \only<4->{\item Mantissa Multiplication}
    \only<5->{\item Right-shifting the mantissa to $[2^{54}, 2^{55})$}
    \only<6->{\item Combining the results and rounding (FPR)}
\only<2->{\end{enumerate}}

\column{0.5\textwidth}
FPN addition (FprAdd) is proceeded by

    \only<7->{\begin{enumerate} \item  Making the first operand $\geq$ the second}
    \only<8->{ \item Right-shifting the second operand}
    \only<9->{ \item Mantissa Addition / Subtraction}
    \only<10->{ \item Normalizing the sum to $[2^{54}, 2^{55})$}
	\only<11->{\item  Combining the results and rounding (FPR)}
\only<7->{\end{enumerate}}

\end{columns}
\medskip

\pause

\end{frame}


\begin{frame}{Sticky Bit}

In floating-point arithmetic, when shifted right, the mantissa maintains a sticky bit
\[
1001\textcolor{black!50!green}{0}\textcolor{red}{0100} \gg 4 \to 1001\underbrace{\textcolor{blue}{1}}_{\text{Sticky}}
\]
It indicates whether there exists any $1$ after the least significant bit. In the above example,
\[
\text{sticky bit} = \textcolor{black!50!green}{0} \vee \llbracket (\textcolor{red}{0100}) \neq 0 \rrbracket = \llbracket (\textcolor{black!50!green}{0}\textcolor{red}{0100}) \neq 0 \rrbracket
\]

\end{frame}

\iffalse
%
%
%
%
%
%
\begin{frame}{Floating-Point Number Packing and Rounding}

\centerline{
\blockalgstart{.6\textwidth}{{\large FPR}}
\begin{algorithm}[H]
  \label{alg:FPR}
  \algsetup{linenosize=\small}
  \begin{algorithmic}[1]
    \REQUIRE Sign bit $s$, exponent $e$, and 55-bit mantissa $z$
    \ENSURE FPN $x$ packed by $s, e, z$
    \STATE $e \gets e + 1076$
    \STATE $b \gets \llbracket e < 0 \rrbracket$
    \STATE $z \gets z \wedge (b-1)$
    \STATE $b \gets \llbracket z \neq 0 \rrbracket $
    \STATE $e \gets e \wedge (-b)$
    \STATE $x \gets ((s \ll 63) \vee (z \gg 2)) + e \ll 52$
    \STATE $f \gets {\tt0XC8} \gg z^{[3:1]} $
    \STATE $x \gets x + f^{(1)}$ \COMMENT{increment if $z^{[3:1]}$ is 011,110 or 111} \\ 
    \RETURN \hskip -4pt $x$
  \end{algorithmic}
\end{algorithm}
\blockalgend
}

\end{frame}


\begin{frame}{Floating-Point Number Multiplication}

\centerline{
\blockalgstart{.9\textwidth}{{\large FprMul}}
\begin{algorithm}[H]
  \label{alg:fpr_mul}
  \algsetup{linenosize=\small}
  \begin{multicols}{2}
  \begin{algorithmic}[1]
    \REQUIRE FPN $x = (sx, ex, mx)$
    \REQUIRE FPN $y = (sy, ey, my)$
    \ENSURE FPN product of $x$ and $y$
    \STATE $s \gets sx \oplus sy$ \label{alg:FprMul:signbit}
    \STATE $e \gets ex + ey - 2100$ \label{alg:FprMul:exponent}
    \STATE $z \gets mx \times my$ \label{alg:FprMul:mantissa}
    \STATE $b \gets \llbracket z^{[50:1]} \neq 0 \rrbracket$
    \STATE $z \gets z^{[106:51]} \vee b$
    \STATE $z' \gets (z \gg 1) \vee z^{(1)}$
    \STATE $w \gets z^{(106)}$
    \STATE $z \gets z \oplus (z \oplus z') \wedge (-w)$
    \STATE $e \gets e + w$
    \STATE $bx \gets \llbracket ex \neq 0 \rrbracket$, $by \gets \llbracket ey \neq 0 \rrbracket$
    \STATE $b \gets bx \wedge by$
    \STATE $z \gets z \wedge (-b)$ \\
    \RETURN \hskip -4pt ${\sf FPR}(s, e, z)$
  \end{algorithmic}
  \end{multicols}
\end{algorithm}
\vspace{-10pt}
\blockalgend
}

\end{frame}


\begin{frame}{Floating-Point Number Addition}

\centerline{
\blockalgstart{.9\textwidth}{{\large FprAdd}}
\begin{algorithm}[H]
  \label{alg:fpr_add}
  \algsetup{linenosize=\small}
  \begin{multicols}{2}
  \small
  \begin{algorithmic}[1]
    \REQUIRE FPNs $x$ and $y$
    \ENSURE FPN sum of $x$ and $y$
    \STATE $d \gets x^{[63:1]} - y^{[63:1]}$
    \STATE $cs \gets d^{(64)} \vee ((1 - (-d)^{(64)}) \wedge x^{(64)}) $
    \STATE $m \gets (x \oplus y) \wedge (-cs)$
    \STATE $x \gets x \oplus m, y \gets y \oplus m$
    \STATE Extract $(sx, ex, mx)$ and $(sy, ey, my)$ from $x, y$, respectively.
    \STATE $mx \gets mx \ll 3, my \gets my \ll 3$
    \STATE $ex \gets ex - 1078, ey \gets ey - 1078$
    \STATE $c \gets ex - ey$
    \STATE $b \gets \llbracket c < 60 \rrbracket$ \label{alg:FprAdd:large_exp_start}
    \STATE $my \gets my \wedge (-b) $ \label{alg:FprAdd:large_exp_end}
    \STATE $my \gets (my \gg c) \vee \llbracket my^{[c:1]} \neq 0 \rrbracket$ \label{alg:FprAdd:unsigned-right-shift}
    \STATE $s \gets sx \oplus sy$
    \STATE $z \gets mx + (-1)^s my $
    \STATE Normalize $z, ex$ to make the 64th bit of $z$ set \label{alg:FprAdd:normalize}
    \STATE $z \gets (z \gg  9) \vee \llbracket z^{[9:1]} \neq 0 \rrbracket$
    \STATE $ex \gets ex + 9$ \\
    \RETURN \hskip -4pt ${\sf FPR}(sx, ex, z)$
  \end{algorithmic}
  \end{multicols}
\end{algorithm}
\vspace{-10pt}
\blockalgend
}

\end{frame}
%
%
%
%
\fi
