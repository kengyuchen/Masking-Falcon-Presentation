\newif\ifhandout
\newif\ifdetailed

%\handouttrue

\ifhandout
\documentclass[handout,xcolor=dvipsnames,aspectratio=169]{beamer}
\else
% \documentclass[xcolor=dvipsnames{beamer}
\documentclass[xcolor=dvipsnames,aspectratio=169]{beamer}

\fi

\usepackage[utf8]{inputenc}
\usepackage{CJKutf8}

% Include Packages
\usepackage{hyperref}
\hypersetup{
    colorlinks=true,
    linkcolor=,
    citecolor=blue
}
\usepackage{mathrsfs}
\usepackage{xcolor}
\usepackage{amsthm,amsmath,amssymb}
\usepackage{stmaryrd}
\usepackage{bm}
\usepackage{hyperref}
\usepackage{graphicx}
\usepackage{float}
\usepackage{subfigure}
\usepackage{caption}
\usepackage{subcaption}
\usepackage{makecell}
\usepackage{multicol}
\usepackage{multirow}
\usepackage{booktabs}
\usepackage{transparent}
\definecolor{trans}{gray}{0.6}
\usepackage{arydshln}

\newcommand{\backupbegin}{
   \newcounter{framenumberappendix}
   \setcounter{framenumberappendix}{\value{framenumber}}
}
\newcommand{\backupend}{
   \addtocounter{framenumberappendix}{-\value{framenumber}}
   \addtocounter{framenumber}{\value{framenumberappendix}} 
}

% \usepackage{algorithm}
% \usepackage{algpseudocode}
\usepackage[noend]{algorithmic,algorithm2e}
\renewcommand{\algorithmicrequire}{\textbf{Input:}}
\renewcommand{\algorithmicensure}{\textbf{Output:}}

\newcommand{\blockalgstart}[2]{
    \begin{minipage}{#1}
    \begin{block}{#2}
    \vskip -8pt
}
\newcommand{\blockalgend}{
    \end{block}
    \end{minipage}
}



\usepackage{tikz}
\usetikzlibrary{shapes.geometric, arrows}
\usetikzlibrary{positioning}
\usetikzlibrary{calc}
\usetikzlibrary{backgrounds}

\tikzstyle{arrow} = [thick,-latex']



\usetheme{CambridgeUS}

% Set Custom Color

\newcommand{\maincolor}{ForestGreen}

\setbeamercolor{palette primary}{fg=black, bg=gray!30!white}
\setbeamercolor{palette secondary}{fg=black, bg=gray!20!white}
\setbeamercolor{palette tertiary}{fg=black, bg=\maincolor!50!white}

\setbeamercolor{title}{fg=\maincolor!60!black}
\setbeamercolor{frametitle}{parent=title}
\setbeamercolor{block title}{parent=title}
\setbeamercolor{caption name}{parent=title}

\setbeamercolor{item projected}{bg=\maincolor}

%%begin novalidate
\newenvironment<>{proposition}[1]{%
  \setbeamercolor{block title}{fg=white,bg=\maincolor}%
  \begin{block}#2{#1}}{\end{block}}
\newenvironment<>{definition}[1]{%
  \setbeamercolor{block title}{fg=white,bg=\maincolor}%
  \begin{block}#2{#1}}{\end{block}}
%%end novalidate
\colorlet{separation rule}{white}

\makeatletter
\pgfdeclareverticalshading[lower.bg,upper.bg]{bmb@transition}{200cm}{%
    color(0pt)=(separation rule); color(2pt)=(separation rule); color(4pt)=(separation rule)}
\makeatother


% Set self-defined words

\DeclareMathOperator*{\argmax}{arg\!\max}
\DeclareMathOperator*{\argmin}{arg\!\min}

\newcommand{\negl}{\mathsf{negl}}
\newcommand{\poly}{\mathsf{poly}}
\newcommand{\Adv}{\mathbf{Adv}}

\newcommand{\R}{\mathbb{R}}
\newcommand{\N}{\mathbb{N}}
\newcommand{\Z}{\mathbb{Z}}
\newcommand{\Q}{\mathbb{Q}}


% Title page configuration
\title[Masking Falcon's FPU]{Masking Floating-Point Number Multiplication and Addition of Falcon}
\author[Keng-Yu Chen, Jiun-Peng Chen]{
Keng-Yu Chen, Jiun-Peng Chen
}
\institute[]{{\small Conference on Cryptographic Hardware and Embedded Systems}}
\date{September 6th, 2024}

%------------------------------------------------------------

\setbeamertemplate{navigation symbols}{}

\AtBeginSection[] % Do nothing for \section*
{
\begin{frame}<beamer>
\frametitle{Table of Contents}
\tableofcontents[currentsection,hideothersubsections ]
\end{frame}
}

\AtBeginSubsection[] % Do nothing for \subsection*
{
\begin{frame}<beamer>
\frametitle{Table of Contents}
\tableofcontents[currentsection, subsectionstyle=show/shaded/hide]
\end{frame}
}
